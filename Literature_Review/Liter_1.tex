\documentclass[UTF8]{article}

\usepackage{graphicx}
\usepackage{setspace}
% \usepackage{ctex}
\usepackage{amsmath}
\usepackage{geometry}


\newtheorem{thm}{Theorem}

\newtheorem{pf}{Proof}
\newtheorem{algorithm}{Algorithm}


% \geometry{a4paper,left=2cm,right=2cm,top=1cm,bottom=1cm}
% \setstretch{1.5}
\geometry{a4paper,scale=0.8}
\renewcommand{\baselinestretch}{1.5}

\title{The literature review of the nine papers}
\author{Dis\cdot count}
% \date{Feb 2019}
\begin{document}

\maketitle{}

\section{Summary}

There are



\section{Some reverse location problems(2000)}

This article discusses about the problem of facilities locations with the fixed locations, and the optimal solution is also not the same.

So it is called \emph{reverse} problem. For the tree network problem, use the minimum cut or maximum flow algorithm (strongly polynomial method) as main subroutine.

\subsection{Literture}

\begin{table}[ht]

\tabcolsep=70pt

\small\renewcommand\arraystretch{2}

\caption{The literature showed in this paper.\label{tab:1}}

{\begin{tabular}{lc}
\hline
Paper & Content \\
\hline
1994 Berman et al. & reverse tree networks problem \\
\hline
1997  &  Inverse matroid intersection problem \\
\hline
1999  &  Inverse polymatroidal flow problem \\
\hline
1998  &  A strongly polynomial algorithm for the inverse arborescence problem \\
\hline
1997 Yang\&Zhang & inverse maximum flow and minimum cut problems \\
\hline
1998 Yang\&Zhang & inverse maximum capacity problem \\
\hline
1998 Zhang\&Cai  & Inverse problem of minimum cuts \\
\hline
1999 Yang\&Zhang & Two general methods for inverse optimization problems \\
\hline
1996 Zhang\&Liu & Calculating some inverse linear programming problem \\
\hline
[17] Zhang\&Liu & Inverse fractional matching problem
\end{tabular}}
{}
\end{table}


\section{Solving Inverse Spanning Tree Problems Through Network Flow Techniques(1999)}

They first study the inverse spanning tree problems which can be transformed to an assignment problem. So just solve the unbalaned assignment problem


\subsection{Literature}


\begin{table}[ht]

\tabcolsep=70pt

\small\renewcommand\arraystretch{2}

\caption{The paper showed in this paper.\label{tab:2}}

{\begin{tabular}{lc}
\hline
Paper & Content \\
\hline
1992\&1994 Burton\&Toint & inverse shortest path problems(L2) \\
\hline
1994\&1995 & some polynomially solvable inverse shortest path problems \\
\hline
1995 Sokkalingam & inverse minimum cost flow problems($L_1,L_2,L_\infty $) \\
\hline
1995 Huang\&Liu & inverse minimum cost flow problem \\
\hline

\end{tabular}}
{}
\end{table}


\section{Inverse Combinatorial Optimization: A Survey on Problems, Methods, and Results(2004)}

This overview should be read carefully.

\subsection{Literture}


\begin{table}[ht]

\tabcolsep=70pt

\small\renewcommand\arraystretch{2}

\caption{The paper showed in this paper.\label{tab:3}}

{\begin{tabular}{lc}
\hline
Paper & Content \\
\hline
1996 &  \\
\hline
1999 & what \\
\hline
1998 & what \\
\hline
\end{tabular}}
{}
\end{table}



\section{Inverse Optimization(2001)}

This article shows many cases.

\subsection{Literture}


\begin{table}[ht]

\tabcolsep=70pt

\small\renewcommand\arraystretch{2}

\caption{The literature showed in this paper.\label{tab:1}}

{\begin{tabular}{lc}
\hline
Paper & Content \\
\hline
1992\&1994 Burton\&Toint & inverse shortest path problems \\
\hline
1996 Zang\&Liu &  inverse assignment and minimum cost flow problems \\
\hline
1997  &  Inverse maximum flow and minimum cut problem \\
\hline
1998  &  Inverse problem of minimum cuts \\
\hline
1997 Yang \& 1998 Zhang\&Cai & inverse minimum cut problems \\
\hline
1995 Xu\&Zhang & inverse weighted minimum cut problems \\
\hline
1999 Sokkalingam \& 2000 Ahuja\&Orlin & inverse spanning tree problem \\
\hline
2001 Ahuja\&Orlin & inverse sorting problem \\
\hline
1998 Ahuja\&Orlin & inverse network flow problems \\
\hline
\end{tabular}}
{}
\end{table}


\section{Inverse Polynomial Optimization(2013)}


This article provides a systematic numerical scheme to compute an inverse optimal solution.

\subsection{Literature}

\begin{table}[ht]

\tabcolsep=70pt

\small\renewcommand\arraystretch{2}

\caption{The paper showed in this paper.\label{tab:1}}

{\begin{tabular}{lc}
\hline
Paper & Content \\
\hline
Condition A & what \\
\hline
Condition B & what \\
\hline
Condition C & what \\
\hline
\end{tabular}}
{}
\end{table}


\section{The inverse optimal value problem(2005)}

This paper considers the following inverse optimization problem:given a linear program, adesired optimal objective value, and a set of feasible cost vectors, determine a cost vector such that the corresponding optimal objective value of the linear program is closest to the desired value.
The above problem, referred here as the inverse optimal value problem, is significantly different from standard inverse optimization problems that involve determining a cost vector for a linear program such that a prespecified solution vector is optimal. In this paper, we show that the inverse optimal value problem is NP-hard in general. We identify conditions under which the problem reduces to a concave maximization or a concave minimization problem.Weprovide sufficient conditions under which the associated concave minimization problem and, correspondingly, the inverse optimal value problem is polynomially solvable. For the case when the set of feasible cost vectors is polyhedral, we describe an algorithm for the inverse optimal value problem based on solving linear and bilinear programming problems. Some preliminary computational experience is reported.


\section{Inverse integer programming(2009)}

Theoretical

We consider the integer programming version of inverse optimization. Using superadditive duality, we provide a polyhedral description of the set of inverse feasible objectives. We then describe two algorithmic approaches for solving the inverse integer programming problem.

We consider inverse integer programming, where an integer vector x0, is given, as well as a constraint matrix, right-handside and a target objective.The goal is to find a vector d that minimizes the weighted norm from a target objective d0 such that x0 is optimal for the pure integer program defined by the objective d. Algorithms for inverse linear programming have been developed and refined in [1] and [8]. The inverse counterparts of various combinatorial optimization problems have been described, including shortest paths, spanning trees, and minimum cost flows. Ahuja and Orlin [1] showed that, under mild conditions, the inverse version of a polynomially solvable optimization problem under the L1 and L∞ norms are polynomially solvable. Less is known about inverse integer programming. Huang[6] showed that the inverse knapsack problem and the general inverse integer programming problem with a  fixed number of rows can be solved in pseudo-polynomial time. See the recent extensive survey of inverse combinatorial optimization by Heuberger [5] for more details.


\subsection{Literature}


\begin{table}[ht]

\tabcolsep=70pt

\small\renewcommand\arraystretch{2}

\caption{The paper showed in this paper.\label{tab:1}}

{\begin{tabular}{lc}
\hline
Paper & Content \\
\hline
Condition A & what \\
\hline
Condition B & what \\
\hline
Condition C & what \\
\hline
\end{tabular}}
{}
\end{table}




\section{Cutting plane algorithms for the inverse mixed integer linear programming problem(2009)}

Theoretical


\subsection{Literture}



\begin{table}[ht]

\tabcolsep=70pt

\small\renewcommand\arraystretch{2}

\caption{The paper showed in this paper.\label{tab:1}}

{\begin{tabular}{lc}
\hline
Paper & Content \\
\hline
Condition A & what \\
\hline
Condition B & what \\
\hline
Condition C & what \\
\hline
\end{tabular}}
{}
\end{table}


\section{Calculating some inverse linear programming problems(1996)}

A method for solving general inverse LP problem including upper and lower bound constraints is suggested which is based on the optimality conditions for LP problems. It is found that when the method is applied to \emph{inverse minimum cost flow problem} or \emph{inverse assignment problem}, we are able to obtain strongly polynomial algorithms.


\subsection{Literature}

\begin{table}[ht]

\tabcolsep=70pt

\small\renewcommand\arraystretch{2}

\caption{The paper showed in this paper.\label{tab:1}}

{\begin{tabular}{lc}
\hline
Paper & Content \\
\hline
Condition A & what \\
\hline
Condition B & what \\
\hline
Condition C & what \\
\hline
\end{tabular}}
{}
\end{table}


\end{document}
