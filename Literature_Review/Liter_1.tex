\documentclass[UTF8]{article}

\usepackage{graphicx}
\usepackage{setspace}
% \usepackage{ctex}
\usepackage{amsmath}
\usepackage{geometry}


\newtheorem{thm}{Theorem}

\newtheorem{pf}{Proof}
\newtheorem{algorithm}{Algorithm}


% \geometry{a4paper,left=2cm,right=2cm,top=1cm,bottom=1cm}
% \setstretch{1.5}
\geometry{a4paper,scale=0.8}
\renewcommand{\baselinestretch}{1.5}

\title{The literature review of the papers about inverse optimization}
\author{Li}
% \date{Feb 2019}
\begin{document}

\maketitle{}

\section{Summary}


\section{Some reverse location problems(2000)}

This article discusses about the problem of facilities locations with the fixed locations, and the optimal solution is also not the same.

So it is called \emph{reverse} problem. For the tree network problem, use the minimum cut or maximum flow algorithm (strongly polynomial method) as main subroutine.

\subsection{Literture}

\begin{table}[ht]

 \tabcolsep = 40pt

 \small\renewcommand\arraystretch{2}

 \caption{The literature showed in this paper.\label{tab:1}}

 {\begin{tabular}{lc}
   \hline
   Paper              & Content                                                              \\
   \hline
   1994 Berman et al. & reverse tree networks problem                                        \\
   \hline
   1997 Cai\&Li       & Inverse matroid intersection problem                                 \\
   \hline
   1999 Cai\&Yang     & Inverse polymatroidal flow problem                                   \\
   \hline
   1998 Hu\&Liu       & A strongly polynomial algorithm for the inverse arborescence problem \\
   \hline
   1997 Yang\&Zhang   & inverse maximum flow and minimum cut problems                        \\
   \hline
   1998 Yang\&Zhang   & inverse maximum capacity problem                                     \\
   \hline
   1998 Zhang\&Cai    & Inverse problem of minimum cuts                                      \\
   \hline
   1999 Yang\&Zhang   & Two general methods for inverse optimization problems                \\
   \hline
   1996 Zhang\&Liu    & Calculating some inverse linear programming problem                  \\
   \hline
   1999 Zhang\&Liu    & Inverse fractional matching problem                                  \\
   \hline
  \end{tabular}}
 {}
\end{table}


\section{Solving Inverse Spanning Tree Problems Through Network Flow Techniques(1999)}

They first study the inverse spanning tree problems which can be transformed to an assignment problem. So just solve the unbalaned assignment problem.


\subsection{Literature}


\begin{table}[ht]

 \tabcolsep = 40pt

 \small\renewcommand\arraystretch{2}

 \caption{The paper showed in this paper.\label{tab:2}}

 {\begin{tabular}{lc}
   \hline
   Paper                    & Content                                                 \\
   \hline
   1992\&1994 Burton\&Toint & inverse shortest path problems(L2)                      \\
   \hline
   1995 Sokkalingam         & inverse minimum cost flow problems($L_1,L_2,L_\infty $) \\
   \hline
   1995 Huang\&Liu          & inverse minimum cost flow problem                       \\
   \hline
  \end{tabular}}
 {}
\end{table}


\section{Inverse Combinatorial Optimization: A Survey on Problems, Methods, and Results(2004)}

Overview



\section{Inverse Optimization(2001)}

This article shows many cases.


\subsection{Literture}


\begin{table}[!h]

 \tabcolsep = 30pt

 \small\renewcommand\arraystretch{2}

 \caption{The literature showed in this paper.\label{tab:4}}

 {\begin{tabular}{lc}
   \hline
   Paper                        & Content                                                                 \\
   \hline
   1992\&1994 Burton\&Toint     & inverse shortest path problems                                          \\
   \hline
   1996 Zang\&Liu               & inverse assignment and minimum cost flow problems                       \\
   \hline
   1997 Yang et al.             & Inverse maximum flow and minimum cut problem                            \\
   \hline
   1997 Yang \& 1998 Zhang\&Cai & inverse minimum cut problems                                            \\
   \hline
   1995 Xu\&Zhang               & inverse weighted minimum cut problems and inverse shortest path problem \\
   \hline
   1999 Sokkalingam             & inverse spanning tree problem                                           \\
   \hline
   2001 Ahuja\&Orlin            & inverse sorting problem                                                 \\
   \hline
   1998 Ahuja\&Orlin            & inverse network flow problems                                           \\
   \hline
  \end{tabular}}
 {}
\end{table}




\section{Inverse Polynomial Optimization(2013)}


This article provides a systematic numerical scheme to compute an inverse optimal solution.

\subsection{Literature}

\begin{table}[ht]

 \tabcolsep = 40pt

 \small\renewcommand\arraystretch{2}

 \caption{The paper showed in this paper.\label{tab:5}}

 {\begin{tabular}{lc}
   \hline
   Paper             & Content                                                                  \\
   \hline
   1999 Huang\&Liu   & inverse linear programming problem and minimum weight perfect k-matching \\
   \hline
   1995 Zhang et al. & column generation method for inverse shortest path problem               \\
   \hline
   2004 Schaefer     & Inverse integer programming(feasible set)                                \\
   \hline
  \end{tabular}}
 {}
\end{table}


\section{The inverse optimal value problem(2005)}

The inverse optimal value problem is NP-hard in general. Under what conditions, the problem reduces to a concave maximization or a concave minimization problem. Under what sufficient conditions the associated concave minimization problem and the inverse optimal value problem is polynomially solvable. For the case when the set of feasible cost vectors is polyhedral, find an algorithm for the inverse optimal value problem based on solving linear and bilinear programming problems.


\subsection{Literature}

\begin{table}[ht]

 \tabcolsep = 40pt

 \small\renewcommand\arraystretch{2}

 \caption{The literature showed in this paper.\label{tab:6}}

 {\begin{tabular}{lc}
   \hline
   Paper                    & Content                                                  \\
   \hline
   1992\&1994 Burton\&Toint & inverse shortest path problems(L2)                       \\
   \hline
   1996\&1999 Zhang\&Liu    & inverse linear programming for the  $l_1, l_\infty$ case \\
   \hline
   1997 Burton et al.       & inverse shortest paths with upper bounds on costs        \\
   \hline
   1999 Fekete et al.       & similar inverse shortest path problem(complexity)        \\
   \hline
  \end{tabular}}
 {}
\end{table}



\section{Inverse integer programming(2009)}

Theoretical

Using superadditive duality, provide a polyhedral description of the set of inverse feasible objectives. Then describe two algorithmic approaches for solving the inverse integer programming problem.


\subsection{Literature}


\begin{table}[ht]

 \tabcolsep = 40pt

 \small\renewcommand\arraystretch{2}

 \caption{The paper showed in this paper.\label{tab:7}}

 {\begin{tabular}{lc}
   \hline
   Paper      & Content                                                         \\
   \hline
   2005 Huang & Inverse knapsack problem(pseudo-polynomial time)                \\
   \hline
   2005 Huang & Inverse integer programming problem with a fixed number of rows \\
   \hline
  \end{tabular}}
 {}
\end{table}




\section{Cutting plane algorithms for the inverse mixed integer linear programming problem(2009)}

Theoretical


\subsection{Literture}


\begin{table}[ht]

 \tabcolsep = 40pt

 \small\renewcommand\arraystretch{2}

 \caption{The paper showed in this paper.\label{tab:8}}

 {\begin{tabular}{lc}
   \hline
   Paper               & Content                                                          \\
   \hline
   1999 Yang\&Zhang    & Column generation and ellipsoid methods for inverse optimization \\
   \hline
   2005 Huang          & Inverse mixed integer and nonlinear programming                  \\
   \hline
   2005 Iyengar\& Kang & Inverse conic programming with applications                      \\
   \hline
  \end{tabular}}
 {}
\end{table}


\section{Calculating some inverse linear programming problems(1996)}

A method for solving general inverse LP problem including upper and lower bound constraints is suggested which is based on the optimality conditions for LP problems. It is found that when the method is applied to \emph{inverse minimum cost flow problem} or \emph{inverse assignment problem}, we are able to obtain strongly polynomial algorithms.


\subsection{Literature}

\begin{table}[ht]

 \tabcolsep=40pt

 \small\renewcommand\arraystretch{2}

 \caption{The paper showed in this paper.\label{tab:9}}

 {\begin{tabular}{lc}
   \hline
   Paper               & Content                                                             \\
   \hline
   1994  Ma\&Xu\&Zhang & Algorithms for inverse minimum spanning tree problem                \\
   \hline
   1995 Xu\&Zhang      & inverse weighted shortest path problem                              \\
   \hline
   1995 Zhang et al.   & inverse shortest path problem with $L_1$ norm and column generation \\
   \hline
  \end{tabular}}
 {}
\end{table}


\end{document}
